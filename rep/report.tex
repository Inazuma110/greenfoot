\documentclass[uplatex]{jsarticle}

\usepackage{mylatex}
\usepackage{ap3}
\usepackage{ascmac}
\usepackage{listings}
\usepackage[dvipdfmx]{graphicx}

\title{第七回応用プログラミング3}
\author{C0117035 伊藤柊太}
\date{5月24日}

\begin{document}
\maketitle

\section{World.javaの説明}
\subsection{コンストラクタ}
14行目で600*400のウィンドウを作る。18行目で、引数に取られているpathの画像をImageIcon型のオブジェクトicon1に設定する。23行目で初期位置(0,0)に画像を配置する。

\subsection{sleepメソッドの説明}
引数tだけなにも処理を行わない。

\subsection{moveObjの説明}
while文で画像の移動を永続的に行う。42行目で画像の現在のx座標を10ずつ移動する。なお、43行目で引数x,yを超えた範囲に移動した際に元の位置に戻す。48行目でsleepにより、100ms動作を停止する。

\begin{lstlisting}[caption=World.java,numbers = left]
import java.awt.*;
import javax.swing.*;

class World extends JFrame {
  JPanel p;
  ImageIcon icon1;
  JLabel label;
  Container contentPane;

  int nx = 0, ny = 0;

  public World(String title){
    setTitle(title);
    setBounds(100, 100, 600, 400);
    setDefaultCloseOperation(JFrame.EXIT_ON_CLOSE);
    p = new JPanel();

    icon1 = new ImageIcon("./pict/pica.png");

    label = new JLabel(icon1);// JlabelにImageIconをセット

    p.setLayout(null);
    label.setBounds(nx, ny, 100, 100);

    p.add(label);// JpanelにJlabelを張り込む

    contentPane = getContentPane();
    contentPane.add(p, BorderLayout.CENTER);


  }

  public void sleep(long t) throws InterruptedException{
    Thread.sleep(t);
  }

  public void moveObj(int x, int y){
    try {
      while(true){
        System.out.println(nx);
        System.out.println(ny);
        nx += 10;
        if (nx > x || ny > y){
          nx = 0;
          ny = 0;
        }
        label.setLocation(nx, ny);
        sleep(100);
      }
    } catch(InterruptedException ie) {

    }
  }
}
\end{lstlisting}

\section{Main.javaの説明}
3行目でWorld型のオブジェクトwを引数"step3"で作成する。5行目で画像を動かす。

\begin{lstlisting}[caption=World.java, numbers=left]
class Main{
  public static void main(String[] args) {
    World w = new World("step3");
    w.setVisible(true);
    w.moveObj(100, 100);
  }
}
\end{lstlisting}

\end{document}
